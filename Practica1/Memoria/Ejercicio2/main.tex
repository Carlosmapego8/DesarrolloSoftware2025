% Inicio del preámbulo

\documentclass[letterpaper,12pt]{article} %Modifica el tipo de documento y el tamaño de la letra.

\usepackage[utf8]{inputenc} %Formato UTF-8 para caracteres especiales.

\usepackage[dvipsnames,table]{xcolor}

\usepackage[shortlabels]{enumitem}
\usepackage[spanish,mexico]{babel} 
\usepackage{amsmath,amssymb,amsfonts,latexsym,cancel}
\usepackage{hyperref}
\usepackage{wrapfig}
\usepackage[rflt]{floatflt}
\usepackage[pdftex]{graphicx}
\usepackage{fancyhdr} %Paquete para el header y el formato de la portada. No sugiero borrarlo!
\usepackage{float}
\usepackage{longtable,multirow,booktabs}
\usepackage{caption}
\usepackage[]{sidecap}
\usepackage{adjustbox}
\usepackage{parskip}
\usepackage{enumitem}
\usepackage{tikz}
\usepackage{lipsum}
\usepackage{wrapfig} %texto alrededor de Images \begin{wrapfigure}{alignment}{width}
\usepackage{xlop}	%divisiones con caja \opidiv{316}{50}
\newcommand\myrule[1]{\multicolumn{1}{| l}{#1}}
\usepackage{multicol} %varias columnas
\setlength{\parindent}{0pt}
\usepackage{textcomp}   %poner símbolo copyright
\usepackage{colortbl}

% bibliografía
\usepackage[nottoc]{tocbibind}
\usepackage[square,numbers]{natbib}
\bibliographystyle{apalike}


% Para hacer recuadros
% https://ondahostil.wordpress.com/2017/05/17/lo-que-he-aprendido-cuadros-de-texto-de-colores-en-latex/
\usepackage{tcolorbox}
\newtcolorbox{mybox}[1]
{colback=RoyalBlue!5!white,colframe=RoyalBlue!75!black,fonttitle=\bfseries,title=#1}

%%% Matemáticas Paquetes AMS
\usepackage{amsfonts}
\usepackage{amsmath} % Matemáticas
\usepackage{amssymb,amsthm} % Símbolos y teoremas
\usepackage{mathtools} % % Algunos añadidos y correcciones a amsmath
\usepackage{upgreek} % Letras griegas
\usepackage{esvect}
\usepackage{dsfont} % Conjunto unario

\numberwithin{equation}{section}
\allowdisplaybreaks[1] % 1-4, permite partir ecuaciones entre páginas

\DeclareMathOperator{\End}{End}
\DeclareMathOperator{\tr}{tr}
\DeclareMathOperator{\Id}{Id}
\DeclareMathOperator{\rg}{rg}
\DeclareMathOperator{\Rad}{Rad}
\DeclareMathOperator{\nul}{nul}
\DeclareMathOperator{\indice}{indice}
\DeclareMathOperator{\Iso}{Iso}
\DeclareMathOperator{\Aut}{Aut}
\DeclareMathOperator{\Gl}{Gl}
\DeclareMathOperator{\im}{Im}
\DeclareMathOperator{\dist}{d}


% teoremas
\theoremstyle{plain}
\newtheorem{teorema}{Teorema}[section]
\newtheorem{prop}{Proposición}[section]
\newtheorem{coro}[teorema]{Corolario}
\newtheorem{lema}[teorema]{Lema}
\theoremstyle{definition}
\newtheorem{definicion}[teorema]{Definición}
\theoremstyle{remark}
\newtheorem*{observacion}{Observación} 
\theoremstyle{remark}
\newtheorem*{ejemplos}{Ejemplos}
\theoremstyle{remark}
\newtheorem*{propiedades}{Propiedades}
\theoremstyle{remark}
\newtheorem*{propiedad}{Propiedad}


\usepackage{multirow} % Tablas con casillas de varias alturas

\definecolor{morado}{HTML}{791FCD}


%Fin de Préambulo

% Variables del documento
% Document Variables 
\newcommand{\myMateria}{Desarrollo de Software}
\newcommand{\myDegree}{Grado en Informática}
\newcommand{\mySemester}{2024-25}
\newcommand{\myReport}{Práctica 1}
\newcommand{\myName}{Cristóbal Merino Sáez, Carlos Manuel Perales Gómez,
Adrián Jaén Fuentes, Kike}
\newcommand{\myUnidad}{}
\date{}



%Inicio formato de Página. 

\textheight = 21cm %Medidas de la  página
\textwidth = 18cm  %Medidas de la página
\topmargin = -2cm  %Medidas de la página    
\oddsidemargin = -0.8cm %Medidas de la página
\pagestyle{fancy} %Diseño de la página

\fancyhf{}
\lhead{\myMateria}%%LeftHead
%%\chead{\includegraphics[ height=1cm]{Images/azalea.jpeg}}%%CenterHead
%\lfoot{USM}
\rhead{\myUnidad}%%RightHead

\setlength{\columnsep}{4mm}%Comandos para el formato de la página.
%\setlength{\parindent}{4em}%Sangría al comenzar un nuevo párrafo.
%\setlength{\parindent}{0.5in}
%\setlength{\parindent}{4em}%Sangría al comenzar un nuevo párrafo.
\setlength{\parskip}{1em}%Distancia entre párrafos.
\renewcommand{\baselinestretch}{1.0}% Espacio entre línea y línea o interlineado.
\setlength{\headheight}{33pt}
\fancyfoot[C,CO]{\thepage} %Logo de LaTeX y pie de página.

%Fin formato de Página

%Aquí inicia el documento.

\begin{document}

    %LaTeX te hace el índice automáticamente conforme añades secciones en tu documento.
    \thispagestyle{empty}
		
		\vspace{0.1cm}
		
		\begin{center}
		
		    \vspace{1cm}
			{\scshape\Large \myMateria \par}
			{\scshape\large Departamento de Lenguajes y Sistemas Informáticos \par}
            \vspace{1cm}

			% Restauramos el interlineado:
			\begin{center}
			
			
			{\large \scshape\myDegree}
			\vspace{0.6cm}
			
			
			\begin{figure}[ht]
			\begin{center}
				\includegraphics[width=0.6\textwidth]{Images/logoUGR.jpg}
				\label{logoUGR}
			\end{center}
				%%\vspace{-1cm}
		\end{figure}	
		
		\LARGE	{\scshape\myReport}
        
		\vspace{1cm}	
        
        \large	 {\scshape\myName}

%% \it es letra itálica
				\vspace{1.25cm}
				\vspace{0.9cm}
				
			\end{center}
	
		\end{center}
    \newpage
    \tableofcontents
    \newpage
%Inicio parte opcional. Esta parte la puedes quitar si deseas, es por si te piden formatos para
%evidencias de certificación de los laboratorios con números de cuenta o te piden abstracts en tus %trabajos.

\title{\myMateria \\\textbf{\myReport} \\ } 

\begin{abstract}
\end{abstract}

\documentclass[a4paper,12pt]{article}
\usepackage[utf8]{inputenc}
\usepackage{hyperref}
\usepackage{graphicx}
\usepackage{listings}
\usepackage{xcolor}

% Configuración de código
\lstset{
    basicstyle=\ttfamily\footnotesize,
    keywordstyle=\color{blue},
    stringstyle=\color{red},
    commentstyle=\color{green},
    breaklines=true,
    frame=single
}

\title{Scraping de citas con Python y el patrón Strategy}
\author{Enrique López García}
\date{\today}

\begin{document}

\maketitle

\section{Introducción}
En esta práctica se ha implementado un programa en Python para extraer información de la web \url{https://quotes.toscrape.com/} utilizando el patrón de diseño \textbf{Strategy}. 
El objetivo es obtener las citas, autores y etiquetas de las primeras 5 páginas de la web, utilizando dos estrategias de scraping: 
\begin{itemize}
    \item \textbf{BeautifulSoup}: Analiza el HTML de la página directamente con la biblioteca BeautifulSoup.
    \item \textbf{Selenium}: Carga dinámicamente la página y extrae el contenido utilizando un navegador automatizado.
\end{itemize}

El programa permite elegir entre ambas estrategias, guarda los datos en un archivo YAML y finaliza cuando el usuario lo decide.

\section{Patrón Strategy e Implementación}
El \textbf{patrón Strategy} permite definir una familia de algoritmos, encapsularlos y hacerlos intercambiables sin modificar el código del programa principal. En este caso, el scraping se implementa con dos estrategias diferentes:

\begin{itemize}
    \item \textbf{Clase Abstracta}: Define la estructura común a ambas estrategias.
    \item \textbf{BeautifulSoupScraper}: Implementa la extracción de datos con la biblioteca BeautifulSoup.
    \item \textbf{SeleniumScraper}: Utiliza Selenium para cargar dinámicamente las páginas y extraer los datos.
    \item \textbf{ScraperContext}: Administra la estrategia seleccionada y coordina la extracción de las páginas.
\end{itemize}

El usuario puede elegir qué estrategia usar pulsando una tecla (`B` para BeautifulSoup, `S` para Selenium, `Q` para salir).

\section{Dependencias y Funciones}
Para la realización de esta práctica se han utilizado las siguientes dependencias:

\begin{itemize}
    \item \textbf{requests}: Para hacer peticiones HTTP a la página web.
    \item \textbf{BeautifulSoup4}: Para analizar y extraer datos del HTML.
    \item \textbf{Selenium}: Para cargar dinámicamente la web y extraer el HTML.
    \item \textbf{PyYAML}: Para guardar los datos extraídos en formato YAML.
    \item \textbf{webdriver-manager}: Para gestionar automáticamente el driver de Selenium.
\end{itemize}

La información, funciones y recomendaciones necesarias para la realización de este ejercicio han sido obtenidas a través de la documentación oficial de las librerías utilizadas, recursos disponibles en Internet y el apoyo puntual de \textbf{ChatGPT} para resolver dudas específicas.

\section{Funcionamiento del Algoritmo}
El algoritmo sigue los siguientes pasos:

\begin{enumerate}
    \item Muestra un menú para seleccionar la estrategia:
    \begin{itemize}
        \item `S` → Selenium
        \item `B` → BeautifulSoup
        \item `Q` → Salir
    \end{itemize}
    
\item Extrae información de las primeras 5 páginas de \url{https://quotes.toscrape.com/}:

\begin{itemize}
    \item En cada iteración, el programa construye la URL correspondiente a cada página con el formato \texttt{https://quotes.toscrape.com/page/\{n\}/}, donde \texttt{n} va del 1 al 5.
    
    \item Si se ha seleccionado la estrategia \textbf{BeautifulSoup}, se utiliza la función \texttt{requests.get()} para descargar el HTML de la página como texto. Luego, se usa \texttt{BeautifulSoup} para parsear el HTML y acceder a los elementos necesarios mediante selectores CSS (como clases).
    
    \item En cambio, si se selecciona la estrategia \textbf{Selenium}, se lanza un navegador (de forma oculta usando modo headless) y se accede a la página cargándola completamente, como lo haría un usuario. 
    Se accede directamente al árbol DOM de la página usando funciones como \texttt{find\_elements(By.CLASS\_NAME, ***)} para ir localizando los elementos correspondientes. 

    \item Cada cita se guarda como un diccionario con tres claves: \texttt{text}, \texttt{author} y \texttt{tags}. Todos los diccionarios se almacenan en una lista que representa todas las citas extraídas.
\end{itemize}    
    \item Guarda los datos en un YAML llamado \texttt{quotes.yaml}.
    
    \item Imprime en consola las primeras 5 citas extraídas.
    
    \item Vuelve a pedir una opción al usuario hasta que pulse `Q` para salir.
\end{enumerate}

\section{Formato de Guardado y Controles del Script}
Los datos extraídos se guardan en un archivo llamado \texttt{quotes.yaml}, en formato diccionario:

\begin{lstlisting}[language=yaml]
- author: "Albert Einstein"
  text: "The world as we have created it is a process of our thinking..."
  tags: ["change", "deep-thoughts", "thinking"]
- author: "J.K. Rowling"
  text: "It is our choices, Harry, that show what we truly are..."
  tags: ["abilities", "choices"]
\end{lstlisting}


\section{Conclusión}

Tras la implementación y evaluación comparativa de ambas estrategias de scraping, se han podido extraer las siguientes conclusiones:

\begin{itemize}
    \item La estrategia basada en \textbf{BeautifulSoup} presenta un rendimiento significativamente superior en términos de velocidad. Esto se debe a que trabaja directamente sobre el contenido HTML estático obtenido mediante peticiones HTTP, sin necesidad de cargar ni renderizar un navegador.
    
    \item Por su parte, la estrategia con \textbf{Selenium} ofrece una mayor versatilidad, ya que permite interactuar con páginas que requieren ejecución de JavaScript u otros elementos dinámicos. No obstante, esta flexibilidad conlleva un mayor coste computacional y un tiempo de ejecución más elevado, debido a la inicialización del navegador y la espera a que el contenido se cargue completamente.
\end{itemize}

En el contexto de esta práctica, dado que la página web analizada no contiene contenido dinámico, la utilización de BeautifulSoup resulta más eficiente y adecuada.


\end{document}


% \newpage

% \glsaddall
% \printglossary

% \newpage

% \nocite{*}
% \bibliography{Libros}

\end{document}

 %Fin del documento.